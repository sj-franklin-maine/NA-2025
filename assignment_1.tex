%%
%% SJ Franklin
%% COS 574 - Professor Quinlan
%% Assignment 1 Template - 2025/09/17
%%

\documentclass{article}
\usepackage{graphicx} % Required for inserting images
\usepackage{amsmath}
\usepackage{amssymb}
\usepackage[T1]{fontenc}
\usepackage{beramono}
\usepackage{listings}
\usepackage[usenames,dvipsnames]{xcolor}

\usepackage[margin=1.8cm]{geometry}
\geometry{a4paper}
\usepackage[parfill]{parskip}

\newlength{\minuslength}
\settowidth{\minuslength}{$-$}

\newcommand{\mh}{\hspace{\minuslength}}
\newcommand{\minus}{\scalebox{0.75}[1.0]{$-$}}

%%
%% Julia definition (c) 2014 Jubobs
%%
\lstdefinelanguage{Julia}%
  {morekeywords={abstract,break,case,catch,const,continue,do,else,elseif,%
      end,export,false,for,function,immutable,import,importall,if,in,%
      macro,module,otherwise,quote,return,switch,true,try,type,typealias,%
      using,while},%
   sensitive=true,%
   alsoother={\$},%
   morecomment=[l]\#,%
   morecomment=[n]{\#=}{=\#},%
   morestring=[s]{"}{"},%
   morestring=[m]{'}{'},%
}[keywords,comments,strings]%

\lstset{%
    language         = Julia,
    basicstyle       = \ttfamily,
    keywordstyle     = \bfseries\color{blue},
    stringstyle      = \color{magenta},
    commentstyle     = \color{ForestGreen},
    showstringspaces = false,
}


\title{Numerical Analysis - Assignment 1}
\author{SJ Franklin}
\date{September 2025}

\begin{document}

\maketitle

\section{Explorations}

\subsection{2.6.9}

Question: Let

\[ A=\begin{bmatrix}{}
2 & -4 \\
3 & -5 \\
7 & 0
\end{bmatrix}
\text{, } \space \space
%
B=\begin{bmatrix}{}
-5 & 1 & 4 \\
\mh4 & 6 & 8 \\
\mh10 & 5 & 0
\end{bmatrix}
\text{, } \space \space
%
C=\begin{bmatrix}
\mh8 & \mh0 \\
\mh2 & -3 \\
-1 & \mh4
\end{bmatrix}
\text{, }
\]

Compute D = A + 3BC by hand, and then use Julia to check your work.

Answer: 

\subsection{2.6.10}

Question: Given a matrix \lstinline[columns=fixed]{A}, write a Julia expression that represents \lstinline[columns=fixed]{A} with its rows and columns both in reverse order, using the \lstinline[columns=fixed]{:} operator.

Answer: 

\subsection{2.7.9}

Question: Modify the scripy \lstinline[columns=fixed]{findmax.jl} from Example 2.7.6 to obtain a new script \lstinline[columns=fixed]{findmaxpos.jl} that not only finds the maximum element of the vector \lstinline[columns=fixed]{x}, but also determines and displays the \textit{index} \lstinline[columns=fixed]{imax}, where $1 \leq \textbf{imax} \leq 10$, at which the maximum value is located.

Answer: 



\subsection{2.7.12}

Question: Estimate the area under the curve $f(x) = x^2, \quad x \in [0,1]$, using a Riemann sum (see Section A.4)
\[
R_n = \sum_{i=1}^n f(x_i^*) \, \Delta x
\]
with step size $\Delta x = 1/n$ using right endpoints $x_i^* = i\Delta x \space \space (i=1,2,\ldots n)$. Repeat for $n=4, 8$ and 16. For each, print the estimate and the absolute difference between the estimate and the exact area, which is $1/3$.


Answer:



\subsection{2.9.1}

Question: Use broadcasting to generate $(\frac{1}{2})^n$ for $n = 1, 2, \ldots, 10$. Then broadcast these values to the square root function, $f(x) = \sqrt{x}$,


Answer:


\subsection{2.9.4}

Question: Let \lstinline[columns=fixed]{x = 1:5}. Use \lstinline[columns=fixed]{map} to perform the following:
\begin{quote}
(a) Rewrite Example 2.9.3 to not use negative exponents, \\
(b) Cube each element of \textbf{x}, \\ 
(c) Raise 3 to the power of \textbf{x}, and \\
(d) Double each element in \textbf{x}.
\end{quote}

Answer: 


\subsection{2.9.5}

Question: Write a single Julia statement that uses \lstinline[columns=fixed]{mapreduce} (or alternatively \lstinline[columns=fixed]{map} and \lstinline[columns=fixed]{reduce}) to evaluate the geometric progression
\[
1+r+r^2+r^3+ \ldots + r^{n-1} = \frac{r^n - 1}{r-1}
\].

Assume that the values $r$ and $n$ are already defined in the global workspace. Use the right side of the equation to check the correctness of your statement.


Answer:



\section{Exercises}

\subsection{Exercise 4}

Question: A ReLU (rectified linear unit) is used as an activation function in deep learning. Write a ReLU function that accepts a numeric value $x$, then determines and prints the result. The (standard) definition of \textbf{ReLU} is:

\begin{equation}
  \text{ReLU}(x) =
    \begin{cases}
      x & \text{if $x \geq 0$}\\
      0 & \text{otherwise .}
    \end{cases}       
\end{equation}


Answer:


\subsection{Exercise 11}

Question:
Use element-wise multiplication on the range \lstinline[columns=fixed]{1:5} and \lstinline[columns=fixed]{fill(1, 5, 5)} to create the following matrices,

\[ \begin{bmatrix}{}
1 & 2 & 3 & 4 & 5 \\
1 & 2 & 3 & 4 & 5 \\
1 & 2 & 3 & 4 & 5 \\
1 & 2 & 3 & 4 & 5 \\
1 & 2 & 3 & 4 & 5
\end{bmatrix}
\text{and}
%
\begin{bmatrix}{}
1 & 1 & 1 & 1 & 1 \\
2 & 2 & 2 & 2 & 2 \\
3 & 3 & 3 & 3 & 3 \\
4 & 4 & 4 & 4 & 4 \\
5 & 5 & 5 & 5 & 5
\end{bmatrix}
\]

Then, apply a logical comparison as a bit mask to extract the lower and upper triangular portion of a square matrix \textit{A} based on definitions found in Section B.10.

Answer:

\subsection{Exercise 12}

Question:

A \textbf{Hilbert matrix} is an $n \times n$ nonsingular matrix with entries
\begin{equation}
h_{ij} = \frac{1}{i+j-1},\ \ i, j = 1, 2, \ldots, n.
\end{equation}

Write a Julia function that generates an $n \times n$ Hilbert matrix given a position integer \textit{n}.

Answer:



\end{document}
